\documentclass{article}

\usepackage[a4paper]{geometry}
\usepackage{hyperref}
\usepackage{listings}
\usepackage{color}
\usepackage{bibentry}
\nobibliography*

\title{My personal C++ Coding guidelines}
\author{Bart Vandewoestyne}

\newcommand{\file}[1]{\texttt{#1}}
\newcommand{\code}[1]{\texttt{#1}}

\newcounter{guideline}
\newenvironment{guideline}[1][]{\refstepcounter{guideline}\par\medskip
   \noindent \textbf{Guideline~\theguideline. #1} \em \rmfamily}{\medskip}

\newenvironment{rationale}{\par\medskip
   \noindent \textbf{Rationale} \em \rmfamily}{\medskip}

\newenvironment{example}{\par\medskip
   \noindent \textbf{Example} \em \rmfamily}{\medskip}

\newenvironment{references}{\par\medskip
   \noindent \textbf{References} \em \rmfamily \begin{itemize}}{\end{itemize}\medskip}

\begin{document}
	
\lstset{language=C++,
	backgroundcolor=\color{white},
	basicstyle=\ttfamily,
	showspaces=false,
	showstringspaces=false,
	frame=shadowbox,
	rulesepcolor=\color{blue}}

\maketitle

\tableofcontents

\section{TODO}

Find good references for the following guidelines:
\begin{itemize}
\item Don't use void as function arguments, simply leave it out???
\item Prefer not to create threads in constructors???
\item Setting pointers to NULL after deleting them.
\end{itemize}

\section{Comments}

TODO: should comments on a class or method go in the header file or in the source file?

\begin{guideline}
Do not systematically refer to bug numbers in the code comments when you fix a bug.
Reason 1: this should be in the commit message, and the commit message must refer to the bug tracking system.
Reason 2: if bug tracking system changes or is not available, these comments are useless.
\url{http://programmers.stackexchange.com/questions/175309/code-maintenance-to-add-comments-in-code-or-to-just-leave-it-to-the-version-con}
\url{http://programmers.stackexchange.com/questions/146823/is-it-good-practice-to-comment-with-issue-number}
\url{http://programmers.stackexchange.com/questions/228093/good-idea-to-put-bug-numbers-in-a-comment-in-the-beginning-of-the-source-file}
\end{guideline}

\section{Warnings}

Compile with as much warnings as you can.

For Visual C++:
\begin{itemize}
	\item Use /Wall instead of /W1, /W2, /W3 or /W4 because this displays all /W4 warnings and any other warnings that are not included in /W4, for example, warnings that are off by default.
	\item Use /WX to treat compiler warnings as errors.
\end{itemize}

\section{Header files}

\subsection{Order of inclusion}
\begin{guideline}
Order all files in groups of descending order of specificity, with each group separated by a blank line, including:
\begin{enumerate}
\item the declaring header(s) file (for implementation files only);
\item other include groups for the given application component;
\item other include groups for the organisation;
\item other include groups for 3rdparty software;
\item standard C++ headers;
\item standard C headers;
\end{enumerate}
Lexicographically order all includes within groups.
\begin{rationale}
The reason for descending order of specificity of groups is to expose hidden
dependencies –- coupling! -– in any of the header files.  The reason for
lexicographical ordering with groups is to make it easier to comprehend each
group’s contents.  The reason for a blank line between groups is obvious: to
delineate one group from another to further aid transparency.
\end{rationale}
\begin{references}
\item See \cite{wilson2016,sutter2004,lakos1996}.
\item \url{http://stackoverflow.com/questions/2762568/c-c-include-file-order-best-practices}
\end{references}
\end{guideline}

\subsection{Include guards}

\begin{guideline}
Include guards should not start with underscores, because such identifiers are officially reserved for the implementation of the compiler and the Standard Library, according to the C++ Standard (ISO/IEC 14882:2003).
\end{guideline}
\begin{itemize}
\item \url{https://en.wikibooks.org/wiki/More_C%2B%2B_Idioms/Include_Guard_Macro}
\item \url{http://stackoverflow.com/questions/17307540/include-guard-conventions-in-c}
\end{itemize}

\subsection{Don't write namespace usings in a header file or before an include}
See C++ Coding Standards, Chapter 59.

\section{Types}

\begin{itemize}
\item Types definitely should have no more than 50 methods.
See 'Working Effectively with Legacy Code', page 245.
CppDepend even warns when there are more than 20 methods.
\item Types should not have more than 20 fields (CppDepend warns for this).
\end{itemize}

Use the new cast operators \lstinline{static_cast}, \lstinline{const_cast}, \lstinline{reinterpret_cast} and \lstinline{dynamic_cast} instead of old-style C-casts.
Reference: TODO

Direct initialization vs copy initialization (difference for fundamental types vs user-defined types).  Personally, for fundamental types, i prefer
\begin{lstlisting}
int x = 0;
\end{lstlisting}
instead of
\begin{lstlisting}
int x(0);
\end{lstlisting}
For user-defined types, the story is different.  See for example \url{http://www.gotw.ca/gotw/036.htm}.

\section{Control statements}

\begin{itemize}
\item Always use braces in if-statements, to avoid confusion.
\item Reduce scope of local variables as much as possible.  See \url{http://refactoring.com/catalog/reduceScopeOfVariable.html} and \url{http://stackoverflow.com/questions/23604699/cppcheck-the-scope-of-the-variable-can-be-reduced-and-loop}.
\end{itemize}

Always use brace in a for or if statement.  Use
\begin{lstlisting}
for (...)
{
    doSomething();
}
\end{lstlisting}
instead of
\begin{lstlisting}
for (...)
    doSomething();
\end{lstlisting}
Reason: avoid confusion and possible future bugs.

\section{Whitespace}

\begin{guideline}
Use
\begin{lstlisting}
for (...)
{
}
\end{lstlisting}
instead of
\begin{lstlisting}
for(...)
{
}
\end{lstlisting}
\end{guideline}


\section{Return statements}

\begin{guideline}
Use
\begin{lstlisting}
return x;
\end{lstlisting}
instead of
\begin{lstlisting}
return (x);
\end{lstlisting}
See \url{http://stackoverflow.com/questions/4762662/are-parentheses-around-the-result-significant-in-a-return-statement} and \url{http://stackoverflow.com/questions/161879/parenthesis-surrounding-return-values}.
Reason: it gives the compiler extra work and is not necessary for clarity, so don't use it.
\begin{rationale}
It gives the compiler extra work and is not necessary for clarity, so don't use it.
\end{rationale}
\begin{references}
\item \url{http://stackoverflow.com/questions/4762662/are-parentheses-around-the-result-significant-in-a-return-statement}
\item \url{http://stackoverflow.com/questions/161879/parenthesis-surrounding-return-values}
\end{references}
\end{guideline}

\section{Virtual functions}

\begin{guideline}
Never call virtual functions during construction or destruction.\\
References: see \cite{meyers2005}, Item 9.
\end{guideline}

\begin{guideline}
Although not strictly necessary, also use the \lstinline{virtual} keyword for your derived-class virtual functions.  It improves readability of the code and `advertises' the fact to the user that the function is virtual.  This is important to anyone further sub-classing the derived class without having to check the base class's definition.
See also \url{http://stackoverflow.com/questions/4895294/c-virtual-keyword-for-functions-in-derived-classes-is-it-necessary}.
Use the \lstinline{virtual} keyword for your derived-class virtual functions.
\begin{rationale}
It improves readability of the code and `advertises' the fact to the user that the function is virtual.  This is important to anyone further sub-classing the derived class without having to check the base class's definition.
\end{rationale}
\begin{references}
\item \url{http://stackoverflow.com/questions/4895294/c-virtual-keyword-for-functions-in-derived-classes-is-it-necessary}.
\end{references}
\end{guideline}

\section{Operator overloading}

\begin{guideline}
Implement binary operators as non-members
\begin{rationale}
This maintains symmetry.
\end{rationale}
\begin{example}
\end{example}
\begin{references}
\item \url{http://en.cppreference.com/w/cpp/language/operators#Canonical_implementations}
\item \bibentry{lippman2012}
\end{references}
\end{guideline}

\section{Exceptions}

\begin{guideline}
Catch exceptions by reference.
\begin{references}
\item \bibentry{meyers1996} item 13
\end{references}
\end{guideline}

\begin{guideline}
Prefer range-for.
See \url{https://youtu.be/xnqTKD8uD64?t=9m7s}
\end{guideline}

\section{Subversion}

\begin{itemize}
	\item Create separate commits for code formatting changes.
\end{itemize}

\section{Unit Testing}

\begin{guideline}
Use the right terminology for your test doubles:
\begin{description}
\item[dummy]
\item[fake]
\item[stub]
\item[mock]
\end{description}
See \url{http://martinfowler.com/articles/mocksArentStubs.html} and \url{http://stackoverflow.com/questions/13698969/what-is-the-difference-between-mocks-stubs-and-fake-objects}
\end{guideline}


\bibliography{references}
\bibliographystyle{plain}
\bibliography{BibTeX-files/books,BibTeX-files/articles}

\end{document}
